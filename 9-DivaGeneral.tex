\chapter{General issues\label{chap:general}}
%------------------------

\lettrine[lines=2, loversize=-0.1, lraise=0.1]{W}{e} describe here the basic requirements in order to perform a simple \diva\, analysis as well as the directory structure of the software. 

More complex applications are described in Chap. \ref{chap:running}.


\minitoc

\newpage % 

Two implementations are available: the Command Line version (CL) and the Graphical User Interface (GUI).

The CL version allows for batch processing and customization, but is not practical for beginners. It is much
better suited for repeated and automated treatment (\textit{e.g.} analysis at various depths or with different data sets). Furthermore it will benefit from latest developments before the GUI version.

As most of the recently developed tools are only available for the CL version, the GUI will be described in a distinct chapter.


%
%Only three files are needed as input:
%
%\begin{enumerate}
%\item a file containing the data (\texttt{data.dat})
%\item the specification of the domain of analysis (\texttt{coast.cont}) and 
%\item the list of the parameters used during the analysis (\texttt{param.par}).
%\end{enumerate}

\begin{center}
\fbox{
\begin{minipage}{0.9\textwidth}
\vspace{.25cm}
\textbf{Convention:} \diva\, works with decimal numbers represented with\quad \textbf{.}\quad  not \textbf{,}\quad

If you use excel for exporting data, try to respect the convention if possible. 
\vspace{.25cm}
\end{minipage}
}
\end{center}


\btips

Use editor \textsl{vi} (see Sec. \ref{sec:vi}) to perform this task.

Simply type:

%\begin{verbatim}
%vi topo.gebco
%:,$s/,/./g
%\end{verbatim}
\etips


\btips

If you do not like \texttt{vi}, you can use the following batch command:
\begin{verbatim}
cat file1 | sed s/,/./g > file2
\end{verbatim}

where: \texttt{file1} is the old file and \\
\hphantom{where:} \texttt{file2} is the file where the replacement has been made.
\etips




\section{Data}
%----------------

The data file contains three (or four) columns: \texttt{ X | Y | value | (relative weight)}.\\
If the number of column is three, the fourth column is assumed to take the value $1$. If there are more than four columns, columns 5 and higher are not used by the software.% (but can be used by the user) .

\begin{figure}[H]
\centering

\parbox{.5\textwidth}{
\begin{footnotesize}
\tt
--------\\
25 25	5\\
25 75 10\\
75 75 5\\
75 25 10\\
--------
\end{footnotesize}
}\parbox{.5\textwidth}{
\includegraphics[width=.45\textwidth]{island_data}
}
\caption{Example of a data file and its graphical representation.}
\end{figure}


\section{Contour\label{contourdiva}}
%--------------------------------------

The contour files are defined this way:
\begin{enumerate}
\item The first line indicates the number of contours in the region of interest (say $M$).
\item The second line tells the number of points in the first contour (say $N_{1}$).
\item The next $N_1$ lines are the coordinates of the points of the first contour. The convention for the contour is that \textsl{the land is on the right when you follow the points successively}. The contour is automatically closed, meaning that the last point of a given contour should be different from the first one. 
\item The following line is the number of points of the second contour (say $N_{2}$).
\item \ldots
\item The last $N_M$ lines are the coordinates of the points of the last contour.
\end{enumerate}


\begin{figure}[H]
\centering 
\parbox{.5\textwidth}{

\begin{footnotesize}
\tt
--------\\
2\\
4\\
0 0\\
100 0\\
100 100\\
0 100\\
4\\
40 40\\
40 60\\
60 60\\
60 40\\
--------
\end{footnotesize}

}\parbox{.5\textwidth}{
\includegraphics[width=.45\textwidth]{island_contour}
}

\caption{Example of a contour file and its graphical representation.}
\end{figure}



\section[Additional points of analysis]{Locations of additional points where analysis is required (optional)}
%--------------------------------------------------------------------------------

The file \texttt{valatxy.coord} is a two-column list of locations where you want the analysis to be performed.  

\begin{exfile}[htpb]
\begin{footnotesize}
\texttt{-5 0\\
-4.7436 0\\
-4.4872 0\\
-4.2308 0\\
-3.9744 0\\
-3.7179 0\\
-3.4615 0} 
\end{footnotesize}
\caption{valatxy.coord\label{ex:valatxy}}
\end{exfile}

If there are more than two columns, columns 3 and higher are not used.

\section{Parameters\label{sec:param.par}}
%-------------------------------------------

In each step of a \diva\, analysis, many parameters can be changed according to the particular case you are treating. If you work with the CL version, these parameters have to be entered in the file \texttt{param.par} (see file \ref{paramfile}). If you work with the GUI version, the parameters are chosen one after each other during the analysis.

Here is a example of a \texttt{param.par} file followed by the description of all the parameters.

\begin{exfile}[htpb]
\begin{footnotesize}
\begin{verbatim}
# Lc: correlation length (in units coherent with your data)
10
# icoordchange (=0 if no change of coordinates is to be performed (...)
1
# ispec (output files required)
3
# ireg (mode selected for background field: 0=null guess;         (...)
1
# xori: x-coordinate of the first grid point of the output
15
# yori: y-coordinate of the first grid point of the output
66
# dx: step of output grid
0.2
# dy: step of output grid
0.05
# nx: number of grid points in the x-direction
300
# ny: number of grid points in the y-direction
300
# valex (exclusion value)
-999.0
# snr signal to noise ratio of the whole dataset
30
# varbak: variance of the background field. If zero,              (...)
0
\end{verbatim}
\end{footnotesize}
\caption{param.par\label{paramfile}}
\end{exfile}



\subsection{\texttt{Lc}}
% ------------------------

Global correlation length used for the analysis (see Sec. \ref{sec:parammeaning} for the physical meaning).

\subsection{\texttt{icoordchange}\label{sec:icoord}}
%---------------------------------------------------


Specifies the desired type of coordinate system:\\

\texttt{icoordchange} \begin{minipage}[t]{.7\textwidth} = $0$ if no change is necessary (data position in kilometers);\\
                                                        = $1$ if data positions are given in degrees;\\
                                                        = $2$ if data positions are given in degrees and your domain extends on a wide span of latitude                                                           (uses a cosine projection);\\
                                                        = $-xscale$ to scale $x$ coordinates by $xscale$ before doing anything ( for vertical                                                              sections)
                      \end{minipage}
                      
                      
\subsection{\texttt{ispec}}
%--------------------------

Four base-values specifies the required error outputs:\\

\texttt{ispec}       = 0\qquad: no error field requested\\
\hphantom{\texttt{ispec}}  = 1\qquad: gridded error field specified by \texttt{xori}, \texttt{yori}, \texttt{nx} and \texttt{ny}; \\
\hphantom{\texttt{ispec}}  = 2\qquad: errors at data locations;\\
\hphantom{\texttt{ispec}}  = 4\qquad: errors at locations listed in \texttt{valatxy.coord}.

Then you can combine these four values to obtain several error outputs:

\examples\\
\texttt{ispec}             = 3 (=1+2)\hphantom{+4} \qquad \begin{minipage}[t]{.7\textwidth}means you want gridded error field as well as errors at data locations;\end{minipage}\\ 
\texttt{ispec}             = 7 (=1+2+4) \qquad means you want the three error files.


For computing errors with the real covariance function, %(Sec. \ref{})
simply multiply the \texttt{ispec} value by $-1$:

\example\\
\texttt{ispec}             = -7 \qquad \begin{minipage}[t]{.7\textwidth}{means you want the three error files computed with the help of the real covariance function.}\end{minipage}


Finally, a \textit{poor man's} error estimate (quick and underestimated error field) is available by adding $+10$ to the \texttt{ispec} value:

\example\\
\texttt{ispec}             = 16 \qquad \begin{minipage}[t]{.7\textwidth}{means you want errors at data locations and at points listed in \texttt{valatxy.coord} computed with the \textit{poor man's} error estimate.}\end{minipage}


\subsection{\texttt{ireg}}

Specification of the background field which is subtracted from the data field (Sec. \ref{sec:backgroundfield}).

\texttt{ireg}             = 0\qquad: no background field is subtracted (assuming data are already anomalies); \\
\hphantom{\texttt{ireg}}  = 1 \qquad: the data mean value is subtracted;\\
\hphantom{\texttt{ireg}}  = 2 \qquad: the linear regression of the data (plane) is subtracted.

\subsection{\texttt{xori/yori, nx/ny}}

\texttt{xori/yori} indicate the coordinates of the first grid point while \texttt{nx/ny} indicate the number of grid points in \texttt{x/y} directions.



\subsection{\texttt{valex}}
%-----------------------------

Exclusion value: value used to fill the output matrix when a point corresponds with land.


\subsection{\texttt{snr}}
%---------------------------

Signal-to-noise ratio of the whole dataset (Sec. \ref{sec:formulation}).


\subsection{\texttt{varbak}}
%------------------------------

Variance of the background field.


%\subsection{Other parameters} 
%%----------------------------
%The GUI gives the user the possibility to chose the value of parameters:
%
%\begin{itemize}
%\item Surface coefficient: controls the shape of the triangular element.
%\item Smooth number: controls smoothing of the mesh elements.
%\end{itemize}

\section{Working directories}
% ------------------------------

The Command Line version is contained in \texttt{diva-\divaversion/divastripped}. Four subdirectories exist inside \texttt{divastripped}:

\begin{enumerate}
\item \texttt{input} contains data, contour(s) and parameters needed by \diva\, to perform an analysis.

\item \texttt{meshgenwork} is the mesh generation working directory.

\item \texttt{divawork} is the working directory of the CL version, where the \diva\, calculation is performed. 

\item \texttt{output} contains user results.

\item \texttt{gnuwork} is used for quick visualisation with gnuplot.

\end{enumerate}
