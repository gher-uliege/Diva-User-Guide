\chapter{Installation of the software}
%-------------------------------

\lettrine[lines=2]{D}{iva} is a software designed to run with any operating system (Microsoft Windows, Linux, Mac OS X). The main steps for the the installation are described in this chapter. A more detailed and up-to-date list of instructions is available on the Installation web page of \diva: \url{http://modb.oce.ulg.ac.be/mediawiki/index.php/Diva_installation}

%%while the Graphical User Interface (GUI) is presented in Section~\ref{sec:guiinstall}. Note that the GUI is not up-to-date.

\minitoc


\section{Download and extract the archive}
%-----------------------------------------

Select a directory on your local disk where you want install the software and download the archive available at \url{http://modb.oce.ulg.ac.be/mediawiki/index.php/DIVA#How_to_get_the_code.3F}.

\begin{lstlisting}[style=Bash]
[charles@gher13 ~]$ cd Software/
[charles@gher13 Software]$ wget http://modb.oce.ulg.ac.be/mediawiki/upload/DIVA/releases/GODIVA_07_2012.tar.gz
\end{lstlisting}

Extract the archive and go in the main directory:
\begin{lstlisting}[style=Bash]
[charles@gher13 Software]$ tar -xvf GODIVA_07_2012.tar.gz
[charles@gher13 Software]$ cd GODIVA_07_2012/
\end{lstlisting}

The directory tree has the following structure: \newpage 
\begin{lstlisting}[style=Bash]
[charles@gher13 GODIVA_07_2012]$ tree -d -L 2
.
|-- DIVA3D
|   |-- bin
|   |-- divastripped
|   `-- src
|-- Doc
`-- JRA4
    `-- Climatology

7 directories
\end{lstlisting}


\begin{itemize}
\item \directory{DIVA3D/bin} contains the executables (\texttt{*.a} files) after compilation. Pre-compiled executables for various operating systems are provided in the sub-folders.
\item \directory{DIVA3D/divastripped} is the main working directory at the 2D level. 
\item \directory{DIVA3D/src} contains the Fortran source code. This is where the compilation has to be done.
\item[]
\item \texttt{Doc} contains the link to publications. For \LaTeX users: you find the corresponding \BibTeX entries in the file \texttt{DivaPublications.bib}.
\item \directory{JRA4/Climatology} is the main working directory at the 3D and 4D levels.
\end{itemize}

%----------------------------------------------
\section{Generation of the binaries }
%----------------------------------------------

There are two possibilities to obtain the binaries: 
\begin{enumerate}
\item Compile the source code.
\item Copy the provided binaries.
\end{enumerate}
The second option is provided for cases where the compilation was not possible, mainly because to missing libraries (e.g., NetCDF). 

\subsubsection{Compilation\label{sec:compilation}}

Go in the source directory 
\begin{lstlisting}[style=Bash]
[charles@gher13 GODIVA_07_2012]$ cd DIVA3D/src/Fortran/
\end{lstlisting}
and edit the configuration file \file{divacompile\_options} for the compilation  according to your machine. 

%Example:
%
%# Name of the fortran compiler (ex: ifort,gfortran,pgi,...)
%compiler=ifort
%# Compilation flags
%flags='-O3'
%# Netcdf library
%nclib=/usr/local/lib/netcdf3ifort/libnetcdf.a
%
%    * if your installation knows the nc-config command, you can use 
%
%# Netcdf library if found by executing nc-config: make sure to use backquote `
%nclib=`nc-config --flibs`
%
%
%    * Run the compilation script: 
%
%divacompileall
%
%and check the log file (compilation.log) 

%\section{Command Line version}
%%-----------------------------
%
%This version is directly usable through a shell in the \texttt{divastripped} directory. 
%
%\begin{figure}[htpb]
%\centering
%\parbox{.65\textwidth}{
%\includegraphics[width=.6\textwidth]{shell}
%}\parbox{.35\textwidth}{
%\caption{\texttt{Msys} shell.\label{shell}}
%}
%\end{figure}
%
%
%\subsection{Windows\label{windowsmsys}}
%%--------------------------------------
%
%You have two possibilities to have a Linux-like shell in your Windows environment:
%
%\begin{enumerate}
%
%\item \texttt{Cygwin}: it can easily be installed (or updated) from \url{http://www.cygwin.com/} by running \texttt{setup.exe} found there. \texttt{Cygwin} is a complete Linux-like environment, thus is recommended to Linux users.
%
%\item \texttt{Msys}: it constitutes the minimal environment under Windows. Its installation is described in the following. 
%
%\end{enumerate}
%
%\subsubsection{\texttt{Msys} installation}
%%-----------------------------------------
%
%If you need to install \texttt{Msys} on your computer:
%\begin{enumerate}
%\item Download the compressed file from\\ 
%\url{http://www.mingw.org/download.shtml}
%%\url{http://prdownloads.sourceforge.net/tcl/msys_mingw8.zip}
%\item Unzip \texttt{msys\_mingw8.zip} (or any recent version) in the chosen directory (let us assume in \texttt{C:/}).
%\item Open the \texttt{msys} shell (Fig. \ref{shell}) by double-clicking on the \texttt{msys} icon (MS-DOS command file) located in the \texttt{msys} folder.
%\end{enumerate}
%
%
%
%\subsection{Linux}
%%-----------------
%
%Under Linux, \texttt{msys} (or \texttt{cygwin}) installation is not required.
%
%
%
%\section{Executables\label{sec:executables}} 
%% ------------------------------------------
%
%
%The final step to achieve the \divainstallation consists in copying the pre-compiled versions of the executables (\texttt{*.a} files) corresponding to your system into \texttt{diva-\divaversion/bin}. They are provided in \texttt{diva-\divaversion/bin/Win32}, \texttt{diva-\divaversion/\-bin/\-Linux} and \texttt{diva-\divaversion/\-bin/\-SGI}.
%
%\example for Windows: open a shell in \texttt{diva-\divaversion} and type:\\
%\textcom{cp -fv ./bin/WIN32/*.a ./bin}
%
%

%
%\subsubsection{\tcltk}
%%---------------------
%
%As \tcltk\, is an interpreted language, it has not to be compiled. Modifications you may have made in \texttt{.tcl} files will be taken into account once you will have launched the \diva GUI.  
%
%
%
%\section{Creation of short-cuts}
%% -----------------------------
%
%Operations described in the present section are optional. The purpose is simply to have short-cut like\\ %\raisebox{-2mm}{\includegraphics[height=.7cm]{logo_diva}} for GUI and\\ 
%\raisebox{-2mm}{\includegraphics[height=.7cm]{logo_diva_cl}} for CL on your desktop.
%
%\subsection{Windows}
%%-------------------
%
%\begin{itemize}
%\item Copy \texttt{diva.ico}, \texttt{divacl.ico}, \texttt{diva.bat}, \texttt{divastripped.bat} and \texttt{.profile} (all found in \texttt{diva-\divaversion/\-inst\-all}) into the \texttt{msys} directory.
%
%
%\item Edit \texttt{.profile} and modify the third line according to your configuration.\\
%\example cd \textcom{c:/diva-\divaversion}
%
%\item With explorer, create short-cut to \texttt{msys/divastripped.bat} and move it to the desktop.
%
%\item Right-click on the desktop-shortcut-Properties:\\
%    in General: use \texttt{diva-\divaversion\, CL} instead of shortcut \ldots as text for the CL version,\\
%    in Shortcut: change icon, browse to \texttt{msys} and chose \texttt{divacl.ico}.
%    
%\end{itemize}
%
%
%%\subsection{Linux}
%%%-----------------
%%\begin{itemize}
%%\item Right-click on your desktop and select \textsl{Create launcher},
%%\item Type =  application,\\
%%name = DIVA (as you wish),\\
%%command = path to the \texttt{wish} executable + path to \texttt{main.tcl}.
%%\item In permission: allow executing file as program.
%%\end{itemize}
%
%
%
%\section{Run a test case}
%% -----------------------
%
%Now that everything is installed, we suggest you to run one of the test case provided in the sub-folders of \texttt{examples}. To this end, simply copy the files into the \texttt{divastripped/input/} directory (for example using command \texttt{divaload}, described in Sect. \ref{sec:divaload}). Open a shell, go in \texttt{./divastripped} and type \textcom{divadress}. If the process is not interrupted, our installation has been done properly. If not, a cause of the problem may be the executable: you may have to recompile the source, as described in Sec. \ref{sec:compilation}.
%
%
%\btips
%When working under Unix, you may have to convert the compilation script (\texttt{divacompile}) and the \textsl{batch} files (files with a name starting with \texttt{diva*} and file \texttt{dbdb2diva} located in \texttt{diva-\divaversion/divastripped}). 
%
%To perform the conversion, use the command \textcom{divados2unix} \footnote{This script uses \texttt{dos2unix}, which should be available within your \texttt{msys} or \texttt{cygwin} distribution.} provided in the \texttt{install} directory.
%\etips
%
%
%\btips
%Depending on the O.S. you have, the command \texttt{dos2unix} may behave differently. In some cases, running \texttt{dos2unix} twice will return you the original file (\textit{i.e.} the first time will convert the file to \texttt{unix}, the second one to \texttt{dos}). In that case you have to specify an option, such as
%
%\textcom{dos2unix -U}
%
%Hence, in case of doubt it is useful to have a look at the \texttt{dos2unix} command description for your system.
%\etips
%
%%\section{PlPlot library}
%%% ----------------------
%%\hypertarget{PLPLOT}{}
%%The PlPlot library is a set of Fortran routines that allow you to obtain graphical outputs without resort to external softwares, such as \matlab\, or NcView. Her is the procedure for the installation:
%%
%%\begin{enumerate}
%%
%%\item download the most recent version of PlPlot from\\
%%\url{http://plplot.sourceforge.net/}
%%
%%\item uncompress and unpack the archive \texttt{plplot-5.7.1.tar.gz}
%%
%%\begin{listevide}
%%\item \textcom{gunzip plplot-5.7.1.tar.gz}
%%\item \textcom{tar xvf plplot-5.7.1.tar.gz}
%%\end{listevide}
%%
%%\item in the directory \texttt{plplot-5.7.1}, type:
%%
%%\begin{listevide}
%%\item \textcom{./configure --prefix=MYPREFIX}, where MYPREFIX stands for the installation prefix (\example \texttt{/usr/local/plplot}) 
%%\item \textcom{make}
%%\item \textcom{make install}
%%\end{listevide}
%%
%%\item in the directory MYPREFIX/share/plplotVERSION/examples, type:
%%\begin{listevide}
%%\item \textcom{make}
%%\item \textcom{./plplot-test.sh} to see some examples.
%%\end{listevide}
%%
%%The complete installation procedure is described in the file \texttt{INSTALL}.
%%
%%\end{enumerate}
%
%

%\section[Installation checking]{Installation checking \expert}
%%-----------------------------------------
%
%This part is quite technical and is not essential for further use of the software. In directory \texttt{diva-\divaversion/install/} you find a series of tools allowing you to perform an elaborated check of your installation by comparing results of \diva executions with your installation with reference outputs. 
%
%Additional information can be found in \texttt{diva-\divaversion/install/README}.
%
%
%\subsection{\texttt{divamakecheck}}
%%----------------------------------
%
%Apply analysis on test cases listed in \texttt{divachecklist} and compare the results with references by applying \texttt{divacomp}.
%The command\\
%\texttt{divamakecheck -generate}\\
%will create new reference fields for later comparisons. Only developers shall use the \texttt{-generate} option.
%
%\subsection{\texttt{divacomp}}
%%-----------------------------
%
%Compare \texttt{ascii} output files generated by a \diva run with reference files provided with the distribution. Results of the comparison are written in \texttt{check.log}.
%
%
%\subsection{\texttt{divados2unix}}
%%---------------------------------
%
%Apply the conversion from dos-like to unix-like line endings on every \diva\texttt{ascii} file and set the permissions of the executables (\texttt{*.a} files) to 755. 
%
%
%
%
