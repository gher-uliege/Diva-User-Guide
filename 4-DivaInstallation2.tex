\chapter{Installation of the software}
%-------------------------------

\lettrine[lines=2]{D}{iva} is a software designed to run with any operating system (Microsoft Windows, Linux, Mac OS X). The main steps for the the installation are described in this chapter. A more detailed and up-to-date list of instructions is available on the Installation web page of \diva: \url{http://modb.oce.ulg.ac.be/mediawiki/index.php/Diva_installation}

%%while the Graphical User Interface (GUI) is presented in Section~\ref{sec:guiinstall}. Note that the GUI is not up-to-date.

\minitoc

\newpage

\section{Requirements}
%---------------------

The basic requirements to run \diva are:
\begin{enumerate}
\item A command-line interface. With Linux or Mac, the interface is directly available: it is the shell or terminal. With Windows, it is necessary to install a Unix-like environment such as Cygwin (\url{http://www.cygwin.com/}).
\item A Fortran 95 compiler, such as:
\begin{itemize}
\item gfortran (\url{http://gcc.gnu.org/wiki/GFortran}),
\item ifort (Intel\textsuperscript{\textregistered}, \url{http://software.intel.com/en-us/intel-compilers}),
\item pgf (Portland Group, \url{http://www.pgroup.com/}).
\end{itemize}    
\item The NetCDF library (\url{http://www.unidata.ucar.edu/software/netcdf/}) for Fortran.
\end{enumerate}

For a quick visualization of the results, a software able to read and display the content of a NetCDF \index{NetCDF} file is recommended:
\begin{itemize}
\item ncBrowse (\url{http://www.epic.noaa.gov/java/ncBrowse/}, a Java application,
\item Ncview (\url{http://meteora.ucsd.edu/~pierce/ncview_home_page.html}), a visual browser, 
\item Panoply (\url{http://www.giss.nasa.gov/tools/panoply/}, the NASA data viewer for various data formats.
\end{itemize}


\section{Download and extraction of the archive}
%-----------------------------------------

Select a directory on your local disk (here we install in a directory \directory{\~{}/Software/}) where you want install \diva and download the archive available at \url{http://modb.oce.ulg.ac.be/mediawiki/index.php/DIVA#How_to_get_the_code.3F}.

\begin{lstlisting}[style=Bash]
[charles@gher13 ~]$ cd Software/
[charles@gher13 Software]$ wget http://modb.oce.ulg.ac.be/mediawiki/upload/DIVA/releases/GODIVA_07_2012.tar.gz
\end{lstlisting}

Extract the archive and go in the main directory:
\begin{lstlisting}[style=Bash]
[charles@gher13 Software]$ tar -xvf GODIVA_07_2012.tar.gz
[charles@gher13 Software]$ cd GODIVA_07_2012/
\end{lstlisting}

The directory tree has the following structure: %\newpage 
\begin{lstlisting}[style=Bash]
[charles@gher13 GODIVA_07_2012]$ tree -d -L 2
.
|-- DIVA3D
|   |-- bin
|   |-- divastripped
|   `-- src
|-- Doc
`-- JRA4
    `-- Climatology

7 directories
\end{lstlisting}


\begin{itemize}
\item \directory{DIVA3D/bin} contains the executables generated by the code compilation. Pre-compiled executables for various operating systems are provided in the sub-folders.
\item \directory{DIVA3D/divastripped} is the main working directory at the 2D level. 
\item \directory{DIVA3D/src} contains the Fortran source code. This is where the compilation has to be done.
\item[]
\item \texttt{Doc} contains the link to publications. For \LaTeX users: you find the corresponding \BibTeX entries in the file \texttt{DivaPublications.bib}.
\item \directory{JRA4/Climatology} is the main working directory at the 3D and 4D levels.
\end{itemize}

%----------------------------------------------
\section{Generation of the binaries (executables)}
%----------------------------------------------

There are two possibilities to obtain the binaries: 
\begin{enumerate}
\item Compile the source code.
\item Copy the provided binaries.
\end{enumerate}
The second option is provided for cases where the compilation was not possible, mainly because to missing libraries (e.g., NetCDF). 

\subsection{Compilation\label{sec:compilation}}
\index{Compilation}
Go in the source directory 
\begin{lstlisting}[style=Bash]
[charles@gher13 GODIVA_07_2012]$ cd DIVA3D/src/Fortran/
\end{lstlisting}
and edit the configuration file \file{divacompile\_options} for the compilation  according to your machine. 

\begin{verbatim}
# Name of the fortran compiler (ex: ifort,gfortran,pgi,...)
compiler=ifort
# Compilation flags
flags='-O3'
# Netcdf library
nclib=/usr/local/lib/netcdf3ifort/libnetcdf.a
...
\end{verbatim}

If your installation knows the \command{nc-config} command, the compiler and the options for the NetCDF library will be detected automatically during the compilation. If you want to check before compilation, type:
\begin{lstlisting}[style=Bash]
[charles@gher13 ~]$ nc-config --fc
gfortran
[charles@gher13 ~]$ nc-config --flibs
-L/usr/lib -lnetcdff -lnetcdf
\end{lstlisting}
Check the options of this command by typing \command{nc-config}, or visit the web page: \url{http://www.unidata.ucar.edu/software/netcdf/workshops/2011/utilities/Nc-config.html}.


Then run the compilation script: 
\begin{lstlisting}[style=Bash]
[charles@gher13 Fortran]$ divacompileall
\end{lstlisting}
and check the content of the log file (\file{compilation.log}). You should obtain something similar to that:
\begin{verbatim}
Compilation time:  Thu Oct 25 14:21:54 CEST 2012
compiler:          ifort
compilation flags: -O3
Calc directory:       1/1   program compiled
Extensions directory: 12/12 programs compiled
Mesh directory:       9/9   programs compiled
NC directory:         3/3   programs compiled
PlPlot directory:     1/1   programs compiled
Util directory:       38/38 programs compiled
Pipetest directory:   1/1   program compiled
Stabil directory:     28/28 programs compiled
----------------------------------------------------------
TOTAL:                93/93 programs compiled
----------------------------------------------------------
Binaries are located in directory:
/home/charles/Software/GODIVA_07_2012/DIVA3D/bin
\end{verbatim}

\section{Direct copy of pre-compiled binaries}

If the compilation failed, go in the \directory{GODIVA\_07\_2012/DIVA3D/bin/} directory and copy directly the binaries available in the sub-directories. For example for Windows with the Cygwin tool:
\begin{lstlisting}[style=Bash]
[charles@gher13 bin]$ cp -f cygwin/* .
\end{lstlisting}

\section{Run tests\label{sec:divatest}}
%------------------

In the main working directory (\directory{GODIVA\_07\_2012/DIVA3D/divastripped}), run one the two available tests: \command{divatest} and \command{divabigtest}. 


\subsection{Basic test}

\command{divatest} creates basic input files (see Chapter~\ref{chap:general}) for details), performs a simple \diva execution and check if the pipes are supported in your operating system (O.S.). The output can be checked using any software for reading NetCDF files. In this case we use ncview (see Chapter~\ref{chap:postprocessing} for details and installation). 

\begin{lstlisting}[style=Bash]
[charles@gher13 divastripped]$ divatest
...
[charles@gher13 divastripped]$ ncview output/ghertonetcdf/results.nc
\end{lstlisting}
The results you obtain has to be similar to those of Fig.~\ref{fig:diva_test_results}.

\begin{figure}[H]
\centering 
\includegraphics[width=.7\textwidth]{diva_test_results}
\caption{Results obtained with \command{divatest}.\label{fig:diva_test_results}}
\end{figure}

\subsection{Large-memory test}
%-----------------------------

\command{divabigtest} creates input files to simulate a case with a large number of data and a very fine mesh. Again, the results are viewable using the command:
\begin{lstlisting}[style=Bash]
[charles@gher13 divastripped]$ ncview output/ghertonetcdf/results.nc
\end{lstlisting}
and should be close to Fig.~\ref{fig:diva_bigtest_results}.

\begin{figure}[H]
\centering 
\includegraphics[width=.7\textwidth]{diva_bigtest_results}
\caption{Results obtained with \command{divabigtest}.\label{fig:diva_bigtest_results}}
\end{figure}



%\section{Command Line version}
%%-----------------------------
%
%This version is directly usable through a shell in the \texttt{divastripped} directory. 
%
%\begin{figure}[htpb]
%\centering
%\parbox{.65\textwidth}{
%\includegraphics[width=.6\textwidth]{shell}
%}\parbox{.35\textwidth}{
%\caption{\texttt{Msys} shell.\label{shell}}
%}
%\end{figure}
%
%
%\subsection{Windows\label{windowsmsys}}
%%--------------------------------------
%
%You have two possibilities to have a Linux-like shell in your Windows environment:
%
%\begin{enumerate}
%
%\item \texttt{Cygwin}: it can easily be installed (or updated) from \url{http://www.cygwin.com/} by running \texttt{setup.exe} found there. \texttt{Cygwin} is a complete Linux-like environment, thus is recommended to Linux users.
%
%\item \texttt{Msys}: it constitutes the minimal environment under Windows. Its installation is described in the following. 
%
%\end{enumerate}
%
%\subsubsection{\texttt{Msys} installation}
%%-----------------------------------------
%
%If you need to install \texttt{Msys} on your computer:
%\begin{enumerate}
%\item Download the compressed file from\\ 
%\url{http://www.mingw.org/download.shtml}
%%\url{http://prdownloads.sourceforge.net/tcl/msys_mingw8.zip}
%\item Unzip \texttt{msys\_mingw8.zip} (or any recent version) in the chosen directory (let us assume in \texttt{C:/}).
%\item Open the \texttt{msys} shell (Fig. \ref{shell}) by double-clicking on the \texttt{msys} icon (MS-DOS command file) located in the \texttt{msys} folder.
%\end{enumerate}
%

%

%
%\section[Installation checking]{Installation checking \expert}
%%-----------------------------------------
%
%This part is quite technical and is not essential for further use of the software. In directory \texttt{diva-\divaversion/install/} you find a series of tools allowing you to perform an elaborated check of your installation by comparing results of \diva executions with your installation with reference outputs. 
%
%Additional information can be found in \texttt{diva-\divaversion/install/README}.
%
%
%\subsection{\texttt{divamakecheck}}
%%----------------------------------
%
%Apply analysis on test cases listed in \texttt{divachecklist} and compare the results with references by applying \texttt{divacomp}.
%The command\\
%\texttt{divamakecheck -generate}\\
%will create new reference fields for later comparisons. Only developers shall use the \texttt{-generate} option.
%
%\subsection{\texttt{divacomp}}
%%-----------------------------
%
%Compare \texttt{ascii} output files generated by a \diva run with reference files provided with the distribution. Results of the comparison are written in \texttt{check.log}.
%
%
%\subsection{\texttt{divados2unix}}
%%---------------------------------
%
%Apply the conversion from dos-like to unix-like line endings on every \diva\texttt{ascii} file and set the permissions of the executables (\texttt{*.a} files) to 755. 
%
%
%
%
